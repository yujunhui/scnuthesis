%%
%% This is file `thesis.tex',
%% generated with the docstrip utility.
%%
%% The original source files were:
%%
%% scnuthesis.dtx  (with options: `thesis')
%%
%% !Mode:: "TeX:UTF-8"
%%
%% This is a generated file.
%%
%% Copyright (C) 2015 by Joseph Pan <cs.wzpan@gmail.com>
%%
%% This file may be distributed and/or modified under the
%% conditions of the LaTeX Project Public License, either version 1.3a
%% of this license or (at your option) any later version.
%% The latest version of this license is in:
%%
%% http://www.latex-project.org/lppl.txt
%%
%% and version 1.3a or later is part of all distributions of LaTeX
%% version 2004/10/01 or later.
%%
%% To produce the documentation run the original source files ending with `.dtx'
%% through LaTeX.
%%
%% Any Suggestions : Joseph Pan <cs.wzpan@gmail.com>
%% Thanks LiuBenYuan <liubenyuan@gmail.com> for the nudtpapre class!
%% Thanks Xue Ruini <xueruini@gmail.com> for the thuthesis class!
%% Thanks sofoot for the original NUDT paper class!
%%
%1. 如果是研究生论文,常用的选项是:
% \documentclass[master,twoside,vista,ttf]{scnuthesis}
%2. 如果是博士生论文,常用的选项是:
% \documentclass[doctor,twoside,vista,ttf]{scnuthesis}
%3. 如果使用是Windows XP之前的Windows系列,或者使用从这个系列拷贝过来的字体,则需要将Vista选项去掉,如:
% \documentclass[master,twoside,ttf]{scnuthesis}
%4. 建议使用OTF字体获得较好的页面显示效果
%   OTF字体从网上获得,各个系统名称统一,不用加vista选项
%   如果你下载的是最新的(1201)OTF英文字体,建议修改scnuthesis.cls,使用PS Std
%   \documentclass[doctor,twoside,otf]{scnuthesis}
%5. 如果想生成盲评,传递anon即可,仍需修改个人成果部分
% \documentclass[master,otf,anon]{scnuthesis}
%6. 让章节标题作为页眉,可以使用chapterhead选项。如果和twoside一起使用,则奇数页页眉为章节标题,偶数页为文章标题。
% \documentclass[master,otf,twoside,chapterhead]{scnuthesis}
%
\documentclass[master,vista,ttf,twoside]{scnuthesis}
\usepackage{myscnu}

\begin{document}
\graphicspath{{figures/}}
% 注意:不需要用到的信息请不要删除, 不然会导致编译不通过, 建议留空即可
\classification{}                   % 分类号, 本科毕业论文留空
\udc{}                                % UDC号, 本科毕业论文留空
\confidentiality{公开}              % 密级, 本科毕业论文留空
\serialno{2011123456}               % 学号
\mastertype{学术}                    % 硕士学位类型(只用于硕士论文), 本科毕业论文留空
\title{华南师范大学硕士/博士学位\\论文\LaTeX{}模板使用手册}                   % 论文中文题目
\entitle{HOW TO USE THE \LaTeX{} DOCUMENT CLASS\\ FOR SCNU DISSERTATIONS}     % 论文英文题目, 本科毕业论文留空
\displaytitle{华南师范大学硕士/博士学位论文\LaTeX{}模板使用手册}              % 论文页眉
\author{张三}                        % 作者中文名
\enauthor{San Zhang}                 % 作者英文名, 本科毕业论文留空
\subject{计算机应用技术}             % 专业中文名
\ensubject{Computer Applications Technology}      % 专业英文名, 本科毕业论文留空
\researchfield{数字图像处理}         % 研究领域, 本科毕业论文留空
\school{计算机学院}                    % 学院名称
\supervisor{李四}                       % 导师中文名
\ensupervisor{Si Li}                    % 导师英文名, 本科毕业论文留空
\protitle{教授}                          % 导师职称, 本科毕业论文留空
\zhdate{\zhtoday}                        % 日期, 本科毕业论文留空
\graduatetime{2018年6月}                % 毕业时间, 只用于本科论文
% 插入摘要,制作封面
\ifisanon{}\else{\maketitle}\fi
\frontmatter
\input{data/abstract}

% 生成目录
\tableofcontents
\listoftables           % 如果要生成表目录
\listoffigures          % 如果要生成图目录

\renewcommand{\chapterlabel}{\denotationname} %设置页眉

\input{data/denotation} % 如果要生成符号列表

% 书写正文,可以根据需要增添章节。
\mainmatter
\input{data/chap01}
\input{data/chap02}
\input{data/chap03}

% 参考文献
\cleardoublepage
\renewcommand{\chapterlabel}{\bibname} % 设置参考文献的页眉
\bibliographystyle{bstutf8}
\bibliography{ref/refs}

% 附录
\appendix
\backmatter
\input{data/appendix01}
\input{data/appendix02}

% 致谢
\cleardoublepage
\renewcommand{\chapterlabel}{\ackname} % 设置参考文献的页眉
\input{data/ack}

% 作者攻读学位期间发表的学术论文目录
\cleardoublepage
\renewcommand{\chapterlabel}{\resumename} % 设置作者个人成果的页眉
\input{data/resume}

\end{document}
%%
